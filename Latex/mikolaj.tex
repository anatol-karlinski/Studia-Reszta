\documentclass[12pt, a4paper]{article}
\usepackage{polski}
\usepackage[UTF8]{inputenc}
\usepackage{graphicx}
\usepackage[export]{adjustbox}
\usepackage{titlesec}
\usepackage{amsmath}
\newcommand{\sectionbreak}{\clearpage}
\author{Anatol Karliński}
\title{\LARGE Jak działa św.Mikołaj}
\date{\vspace{-0ex}}

\begin{document}
\maketitle
\includegraphics[center]{logo.jpg}
\tableofcontents
\section{Fizyka św.Mikołaja}
\subsection{Ilosć wizyt}
Istnieje około biliona dzieci\footnote{Wszystkie dane dotyczące ludnoci pobrane zostały ze strony www.worldometers.info} (osób po niżej wieku 18 lat) na świecie. Wykluczając dzieci z religii nie wierzących w św. Mikołaja zostajemy z ok. 378 milionem dzieci. Zakładając, że na jeden dom przypada 3.5 dziecka, otrzymujemy 108 miliona domów, które muszą zostać odwiedzone.\newline
\par
Ze względu na na strefy czasowe i rotację globu Mikołaj posiada około 31 godzin na wykonanie swojej pracy ( zakładając, że będzie poruszać się ze wschodu na zachód). 31godzin to dokładnie 111.600 sekund.\\
\begin{center}
W związku z powyższym otrzymujemy:
\begin{gather*}
\frac{108*10^6}{111.6*10^2}  = 967.7 
\end{gather*} 
\end{center}
Czyli Mikołaj ma do wykonania 967.7 wizyt w każdej sekundzie,
co daje jedną tysięczną sekundy na dom.
\subsection{Prędkoć}
\par
Jeśli przyjmiemy średnią odległość między odwiedzanymi domostwami za 0.78 mili, to Mikołaj ma do przebycia 75.5 milionów mil. Jeśli nie policzymy przerw na toaletę i odpoczynek, otrzymamy:
\begin{center}
\begin{gather*}
\frac{75.5*10^6}{1116*10^2} = 676.5
\end{gather*} 
\end{center}
Czyli Mikołaj porusza się z prędkocią ok. 676.5 mil na sekundę - czyli 3000 razy szybciej niż prędkoć dzwięku.
\section{Dieta św.Mikołaja}
\subsection{Kalorje}
W trakcie swojej podruży Mikołaj tradycyjnie zjada pozostawione dla niego przekąski takie jak ciastka, cukierki czy mleko.\\
\newline
Oto tabela wartości odżywczych dla najczęściej zostawianych dla Mikołaja słodyczy:
\begin{center}
\begin{tabular}{| l | l | l |}
\hline
 Słodycz & Wartoć kaloryczna na 100g\\ \hline
Ciasko Oreo & 471 kcal\\ \hline
Piernik & 366 kcal\\ \hline
Herbatnik w czekoladzie& 462 kcal\\ \hline
Bajaderka & 317 kcal\\ \hline
Sernik & 273 kcal\\ \hline
Czekolada orzechowa & 520 kcal\\ \hline
\end{tabular}
\end{center}
\par Średnia wartosć kaloryczna słodyczy to 401.5 kcal. 
Pojedyńcze ciasko waży srednio 10g, w związku z tym przeciętny słodycz będzie posiadać 40.1 kcal.\\
Jeśli przyjmiemy, że Mikołaj dla grzeczności zje średnio chociaż jedno ciastko w każdym domu to możemy policzyć, że:
\begin{gather*}
 40kcal*108*10^6 = 4320*10^6
\end{gather*} 
Mikołaj konsumuje ok. 4320 milionów kcal w trakcie swojej podróży. Na całe szczęście spala chociaż część tych kalorii wspinając się na kominy.

\begin{thebibliography}{9}
\addcontentsline{toc}{section}{Literatura}
\bibitem{ilewazy} 
Serwis "Ile Waży" - 
\textit{http://www.ilewazy.pl}. 
 
\bibitem{worldmeters} 
Serwis World Meters - 
\textit{http://www.worldometers.info}. 
 
\bibitem{baltimoremend} 
Artykuł "SANTA CLAUS: An Engineer's Perspective" rok 1998 - 
\textit{http://www.baltimoremd.com/humor/santaengineer.html}. 
\end{thebibliography}

\end{document}